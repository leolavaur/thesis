\section{Discussion\label{sec:radar.discussion}}

The experiments illustrate how \thecontrib succeeds at identifying attackers in heterogeneous context, thus demonstrating its versatility.
In this section, we discuss the limitations and potential consequences of our architecture and propose research directions to close these gaps. 


\paragraph{Heterogeneity}

The experiments conducted in \Cref{sec:radar.results} show that \thecontrib can handle heterogeneous participants.
However, the simulation of the \emph{practical} \gls{niid} setting is limited by the available datasets and the partitioning choices.
In \thecontrib, we use \gls{iid} partitioning with each of our datasets to create our communities.
This choice is motivated by the absence of control over the data distribution of the participants with datasets from different sources.
The approach presented in \Cref{chap:topologies} is expected to cope with this limitation, and thus represent an interesting research direction to further investigate heterogeneous settings.


\paragraph{Generalizability}

While the experiments are only conducted on intrusion detection datasets, \thecontrib's design could be used in different use cases regarding the following conditions: (1) parametric local models whose parameters can be aggregated using \gls{fl}, and (2) local testing sets and relevant metrics allowing participants to evaluate the others’ models.
Since the \gls{nids} use case induces a focus on malicious samples (\ie \emph{positive} values), we choose the F1-score as input for our clustering and reputation algorithms, as it emphasizes on false positives and false negatives.
However, \thecontrib can handle different metrics, for instance the loss of a model during evaluation, particularly relevant for similarity measurements.


\paragraph{Scalability and performance}

The focus on small-scale collaboration (\ie a few dozens of participants) makes the overhead of the \emph{cross-evaluation} step (\Cref{sec:radar.archi.xeval}) practical, and justifies the absence of performance-related metrics in this paper.
However, one can question the scalability of the proposed approach in larger scale applications.
Indeed, at each round, clients evaluate $|P|$ additional models, which scales linearly with the number of clients.
Two new communications are also introduced, one to send the models and one to collect the evaluations.
Their size also grows linearly with $|P|$, as the models of all participants must be evaluated.
Likewise, we exclude execution-related performance evaluation such as training time, CPU overhead, or bandwidth consumption.
It opens the way to interesting research directions on how to implement and scale \thecontrib while guarantying its properties.


\paragraph{Evaluation poisoning}

Attackers could try to poison the evaluations that they provide on other participants to abuse the system.
However, the implementation presented in \Cref{sec:radar.methodo} implies that attackers poison both their training and testing sets.
Consequently, the evaluations they produce on other participants are directly affected.
We thus expect the system to cope with arbitrary poisoning similarly to data poisoning: either by placing the attackers in a different cluster because of their dissimilarity, or by penalizing their reputation. 


\paragraph{Information disclosure}

Because \thecontrib shares models with the other participants to obtain feedbacks, it can be argued that it revels more information about the participants.
This is limited to the participants' models, which are shared without identifiers.
However, since clients also receive the global model of their cluster, they can try to estimate the models that belong to their cluster.
This remains challenging, as the models are weighted using the reputation score of the participants, which are only available to the server.
Comparing the privacy impact of \thecontrib with those of simpler approaches like \texttt{FedAvg} represents interesting research directions.
