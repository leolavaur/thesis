\section{Problem Statement\label{sec:radar.problem}}

In continuity with \Cref{chap:app,chap:assessment}, we consider once more the use case introduced in \Cref{sec:app.overview} and the associated datasets.
Specifically, we focus on a heterogeneous declination of this \gls{cids} use case, where we admit that participants share similarities in their data distributions---\eg, between organizations operating in the same sector or having similar network infrastructure.
This setting, also mentioned in \Cref{sec:app.overview}, is referred to as \emph{practical} \gls{niid}~\cite{huang_PersonalizedCrossSiloFederated_2021}.
We also set $C=1$, as we consider that the participants are highly available and interested in collaborating.

\subsection{Low-quality Contributions\label{sec:radar.problem.quality}}

In \gls{fl}, the quality of the global model is directly impacted by the quality of the participants' contributions.
In a \gls{ids} context, the poor quality of a \gls{ml} model can be induced by some choices in terms of architecture, hyperparameters, or optimizer---all fixed by the server, but also by the quality of the training data.
Multiple factors can affect the quality of local training data~\cite{jain_OverviewImportanceData_2020}, such as: 
\begin{enumerate*}[(1)]
  \item \emph{Label noise}---samples associated with the wrong labels;
  \item \emph{Class imbalance}---differences in terms of class representation in the dataset; or
  \item \emph{Data heterogeneity}---the variations between samples of the same class.
\end{enumerate*}

Similar to existing works on data-quality~\cite{deng_FAIRQualityAwareFederated_2021,deng_AUCTIONAutomatedQualityAware_2022}, we focus on label noise, which can have significant consequences on the global model's performance, depending on the proportion of mislabeled samples.
In a \gls{cids}, label noise can unknowingly be introduced by the participants, either due to misconfigurations or to the presence of compromised devices.
We consider two types of label noise: \emph{missed intrusions} and \emph{misclassification}.
\begin{enumerate}[a)]
  \item \emph{Missed intrusions} occur when a malicious sample is mislabeled as benign, leading to a false negative. Participants in \glspl{cids} label the attacks they are aware of, but some might have been unnoticed.
  \item A \emph{misclassification} is the random mislabeling of a sample. This can be due to a lack of knowledge or to a misconfiguration.
\end{enumerate}
Such participants are referred to as \emph{honest-but-neglectful}.
Because these errors are assumed to be unintentional, the proportion of \emph{misclassified} samples is expected to be low.
However, the concept of \emph{missed intrusions} implies that the participants are not aware of an entire attack, which can represent a significant proportion of their dataset.


\subsection{Data Poisoning Attacks\label{sec:radar.problem.poisoning}}

In addition to accidental low-quality contributions, some participants might deliberately upload model updates that would negatively impact the performance of the global model.
Specifically, we consider the same attack model as detailed in \Cref{chap:assessment}, and focus on label-flipping attacks.
The model can be summarized as follows:
\begin{description}[font=\normalfont\itshape, leftmargin=2em]
  \item[Attackers' Knowledge.] Attackers are \emph{gray-box} adversaries, meaning that they have access to the same information as the other participants; \eg, the last global models, the hyperparameters, or the optimizer.
  \item[Attackers' Objective.] With targeted poisoning, attackers aim at making a specific type of attack invisible to the \gls{nids}.
  Conversely, with untargeted attacks, they seek to jeopardize the \gls{nids} performance by maximizing the number of misclassifications.
  \item[Attackers' Capabilities.] Attackers can flip the labels of an arbitrary proportion of their dataset, referred to as the \gls{dpr} and denoted $\alpha$.
  They can act alone or in collusion with other by applying the same strategy.
  The proportion of attackers in the system is described by the \gls{mpr} and denoted $\beta$. 
\end{description}

Because we do not make a priori assumptions on the whether the participants are malicious or not in this contribution, we also refer to the \gls{dpr} as the \emph{noisiness} of a participant.
The \gls{mpr}, on the other hand, almost exclusively describes attackers, as it is unlikely for the same Byzantine fault to occur in multiple participants simultaneously.

\subsection{Problem Formalization\label{sec:problem.formalization}}

Based on the previous assumptions, we consider that participants might upload model updates that would negatively impact the performance of the global model, deliberately or not.
Multiple forms of such actors can exist: external actors altering legitimate clients' data (\ie \emph{compromised}), clients whose local training sets are of poor quality (\ie \emph{honest-but-neglectful}), or clients modifying their own local data on purpose (\ie \emph{malicious}).
We refer to them as \emph{Byzantine participants} or simply \emph{Byzantines} in the remaining of this paper.

We further consider that the server can be trusted to perform the aggregation faithfully, and that \gls{fl} guaranties the confidentiality of the local datasets.
Attacking the server is out of the scope of this contribution.
Consequently, we aim at weighting or discarding the participants' contributions based on their quality to guaranty the performance of the aggregated model.


% \begin{problem}[Quality Assessment in Heterogeneous Settings]
%   For $n$ participants $p_k$ and their local datasets $d_k$ of unknown similarity, each participant uploads a model update $w_k^r$ at each round $r$. Given $P = \lbrace p_1, p_2, \dots, p_n \rbrace$ and $W = \lbrace w_1^r, w_2^r, \dots, w_n^r \rbrace$, how can one assess the quality of each participant's contribution $w_k^r$ without making assumptions on the data distribution across the datasets $d_k$?
%   \label{pb:qahs}
% \end{problem}

\begin{problembox}{Quality Assessment in Heterogeneous Settings}{qahs}
  For $n$ participants $p_i$ and their local datasets $d_i$ of unknown similarity, each participant uploads a model update $w_i^r$ at each round $r$. Given $P = \lbrace p_1, p_2, \dots, p_n \rbrace$ and $W = \lbrace w_1^r, w_2^r, \dots, w_n^r \rbrace$, how can one assess the quality of each participant's contribution $w_i^r$ without making assumptions on the data distribution across the datasets $d_i$?
\end{problembox}