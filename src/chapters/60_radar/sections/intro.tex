\section{Introduction\label{sec:radar.intro}}

In the previous chapters, we identified and studied two major challenges that currently impede the adoption and deployment of \glspl{fids}: \begin{enumerate*}[(1)]
  \item the heterogeneity of the data sources, notably in \gls{csfl} settings; and
  \item the susceptibility of \glspl{fids} to adversarial attacks.
\end{enumerate*}
More generally, because collaborative systems are inherently sensitive to input quality, any form of Byzantine failure should be considered.
While we focus specifically on data-related failures in the context of this thesis, Byzantine faults can also encompass other types of failures, such as crashes, arbitrary behavior, or communication issues.
This applies whether the participants are honest but use faulty data, or actively malicious.
In this heterogeneous context, it is particularly challenging to distinguish a faulty or malicious contribution from a legitimate one originating from a different type of infrastructure.

Approaches that assess model quality~\cite{pejo_QualityInferenceFederated_2023} or mitigate poisoning~\cite{blanchard_Machinelearningadversaries_2017,cao_FLTrustByzantinerobustFederated_2022} in homogeneous distributions typically compare or evaluate a model using a single source of truth.
Building such a single source of truth, however, is inadequate in heterogeneous contexts due to the differences between participants. 
Assuming that all contributions are therefore different, some approaches detect colluding attackers based on their similarity~\cite{fung_LimitationsFederatedLearning_2020,awan_CONTRADefendingPoisoning_2021}. 
Nevertheless, these approaches fail to detect isolated attackers.

In this chapter, we present \thecontrib, an architecture for \gls{csfl} guarantying high-quality model aggregation, regardless of the data homogeneity.
\thecontrib relies on three main ingredients:
\begin{enumerate*}[label=\em {\roman*})]
    \item a modified \gls{fl} workflow, where each participant uses its local dataset to evaluate the other participants' models, between the training and aggregation steps;
    \item a clustering algorithm leveraging the participants' perceived similarity to aggregate group-specific global models; and
    \item a reputation system that weights the participants' contributions based on their past interactions.
\end{enumerate*}

We evaluate the performance of \thecontrib in a realistic \gls{cids} use case, using four network flow datasets with standardized features, representing different environments, and model various Byzantine behavior using label-flipping.
We also compare our approach to existing strategies~\cite{mcmahan_Communicationefficientlearningdeep_2017,fung_LimitationsFederatedLearning_2020}, and conclude that \thecontrib can detect Byzantines contributions under most scenarios, from noisy labels to colluding poisoning attacks. 

The content of this chapter is based on our work published in \gls{srds}~\cite{lavaur_radar_2024}, which results from a collaboration with Pierre-Marie Lechevalier, another Ph.D. student at IMT Atlantique.
The remainder of this chapter is organized as follows.
We start by introducing preliminary concepts in \Cref{sec:radar.prelim}, before formalizing the problem in \Cref{sec:radar.problem}.
After reviewing related works in \Cref{sec:radar.related}, we dive in \thecontrib's architecture in \Cref{sec:radar.archi}.
\Cref{sec:radar.methodo,sec:radar.results} present the experimental setup and results, that we then discuss in \Cref{sec:radar.discussion}.
Finally, \Cref{sec:radar.conclusion} concludes this chapter.

\begin{contribs}
  \item \thecontrib, an architectural framework to protect \gls{fl} strategies using clustering and reputation-aware aggregation, validated by extensive evaluation against relevant baselines;
  \item Demonstration that evaluation metrics (such as accuracy, F1-score, or loss) can be used to effectively assess similarity between \gls{fl} participants, and as an input to clustering and reputation algorithms;
  \item Confirmation that combining reputation and clustering successfully addresses the problem of contribution quality assessment in heterogeneous settings.
\end{contribs}
