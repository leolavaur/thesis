Words will fall short in expressing the infinite gratitude I feel towards all the people without whom this work would not have been possible.
I have read too many thesis acknowledgments to not be aware of how challenging this exercise is, particularly when striving for the much-desired completeness.
Knowing that it will never be fully achieved, I nonetheless hope to avoid too many oversights, and that the people concerned will recognize themselves in these few lines.

First of all, I would like to warmly thank Vincent Nicomette for accepting to chair my jury, as well as Sonia Ben Mokhtar and Pierre-François Gimenez for agreeing to be a part of it.
A special thanks goes to the reviewers, Anne-Marie Kermarrec (now my academic \emph{grandmother}) and Éric Totel, for agreeing to read and assess a manuscript that may not be the shortest, and that they received during the summer.
These thanks naturally extend to the members of my thesis advisory committee, who followed my work and advised me at each of our meetings: Nur Zincir-Heywood, Hanan Lutfiyya, and David Espes.
Thanks also to my teachers at ENSIBS, especially to Vianney, Seb, and Jamal, for encouraging me to pursue research and providing me with the foundations necessary to succeed.
I also wish to thank Marc-Oliver Pahl, holder of the CyberCNI chair, for recruiting and supervising me, as well as all the partners who contributed to funding this adventure.

This work would not have been possible without my thesis supervisor, Yann Busnel, and my co-supervisor, Fabien Autrel.
Thank you, Yann, for trusting me and giving me total freedom in my work, while guiding and advising me at every stage, despite the distance and your ever-growing professional responsibilities.
Your intuition, experience, and kindness have led me to where I am today.
All of this, of course, without forgetting our shared taste for malted liqueurs.
Thank you, Fabien, for your unconditional daily support, as well as your technical investment without which some parts of this work would never have materialized.
Thank you for the good humor and the jokes of varying quality, but always appreciated.
To both of you, I owe you so much.

Some PhD students see their host institution as just a place to pass through, but the Rennes campus of IMT Atlantique and particularly the SRCD department have been much more than that for me.
Here, I have had such rich and varied discussions that have gone well beyond the professional and scientific sphere.
With Loutfi, about mountains, or French music and cinema if Xavier is around, with Guillaume about cooking and De Buyer frying pans, with Jean-Marie about bikes, with Renzo about philosophy, with Nico about video games\dots{}
Not to mention Hélène, Laurent, Patrick, Alexander, Léonie, Briac, Baptiste, Alexis, Benoit, Georgios, or Nicolas, and those who carry the department and campus on their shoulders: Sandrine, Catherine, Frank, Vincent, and the Delphines.
But in this incredible crowd, I would particularly like to thank this core trio formed by Gégé, Nico, and Rom, who welcomed me, advised me, and taught me so much, while also dragging me into the worst traps of Saint-Hélier.
Thank you all for these moments of sharing and conviviality; I will always cherish these years spent at IMT Atlantique.

My thanks also go to my colleagues within the SOTERN team, in which I have not yet mentioned Ahmed, Pierre, Aymen, Gilles, as well as all the non-permanent members who contributed to the great atmosphere and productivity of the group: Loïc, Antoine, Tien, Anh, Mathis, and the newcomers Hugo, Constant, Dorian, and Gabriel.
Although spread across two sites, I am glad to have worked with such a rich and human team.

Special thoughts go to the PhD and post-doctoral colleagues who shared all or part of this journey with me.
Those who show that it's possible, Renzo and Loïc, my office mates, Gwen and Awaleh, then Arnol, and finally Tien and Loïc again, but also Juliette, Modou, and all the others.
A special thank you to my ENSIBS cartel infiltrated within IMT Atlantique, Antoine and Pierre-Marie, with whom I am proud to form that infernal trio capable of bringing down the school's network.
Pierre-Marie in particular, thank you for putting up with me during these two years of collaboration, even if it meant spending evenings fighting over \LaTeX{} macros or the organization of RADAR's Gitlab.
I am proud of what we accomplished together.

I cannot conclude these acknowledgments without mentioning my friends and family, who supported and encouraged me throughout these years.
My gratitude goes to my \texttt{/(Banana|Sticks)~Team/g}, Mathieu, Nathan, and Hippo, as well as the other "Copains", Kraft, Popo, and Lolo, for the moments of relaxation and laughter that helped me navigate this sometimes difficult period.
Thanks to Brieuc, my brother in heart and friend for life, and his partner Clarisse, for their presence even from a distance, as well as all the others who will recognize themselves.
Thanks to my parents, who have always supported me in my choices and gave me the means to achieve my ambitions, as well as to my sister and all my loved ones who encouraged and supported me, even if they don't always know what I do.
Finally, I am immensely grateful to my partner Emma, my \emph{Mimie}, for her patience, support, and unconditional love, without which I would never have had the strength to finish this work.

Thanks also to all those I haven't mentioned here but who contributed in their own way to the completion of this PhD.
To all, thank you.