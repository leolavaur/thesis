\section{A Practical Use Case for FIDSs\label{sec:app.overview}}
% organizations
% sake of simplicity -> hfl, same features
% not unrealistic, 
% - eg. same probe deployed in multiple organizations
% - gray-box product that all organizations use

We consider a typical \gls{fl} scenario where a central server $S$ is tasked with aggregating the model updates $w_k^r$ of a set of participants $p, k\in \llbracket 1,n \rrbracket$ at each round $r$.
Participants are entities that oversee an organization's network, which makes them highly available and interested.
This can be described as a \gls{csfl} scenario, \ie, fewer participants with consequent amounts of data and significant computing capabilities.
Because of the lower scale of the federation and the assumed interest of the different parties, we set the fraction $C$ of participants that are selected at each round to $1.0$.

For the sake of simplicity, we consider that all participants share the same model architecture and extract the same features from the network traffic.
This is not unrealistic, as common formats and protocols are used in the industry, such as the NetFlow format~\cite{rfc3954} for network flows.
Further, this description can fit multiple scenarios, such as organizations deploying the same probe in their network as part of a standardization effort, or a service provider offering a gray-box product to multiple organizations.

We also consider that participants have access to labeled data, which is a common assumption in the literature.
Although labeling data can be costly, it is a more reasonable assumption in \gls{csfl} scenarios, where participants are more likely to have the human and financial resources to label data.
Therefore, each participant possess a local dataset $d_k = (X_i, Y_i)$ that is not shared with the others.
Because organizations in \gls{cids} may have different network configurations~\cite{zhou_surveycoordinatedattacks_2010}, the distribution of each local dataset $d_k$ can vary considerably, independently of the associated labels.
However, the \gls{cids} use case implies that similarities can exist between participants, for instance between organizations operating in the same sector or having similar network infrastructure.
This particular setting can be described as \emph{practical \gls{niid}}, as opposed to the \emph{pathological \gls{niid}} settings, where all participants have unique and highly different data-distributions~\cite{huang_PersonalizedCrossSiloFederated_2021}.