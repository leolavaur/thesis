\section{Conclusion and Takeaways\label{sec:assess.conclusion}}

The literature on the impact of poisoning attacks on \gls{fl} that specifically cover intrusion detection use cases is scarce, and in it, label-flipping attacks have been overlooked.
This chapter fills this gap by providing a comprehensive analysis of the impact of label-flipping attacks on \gls{fl} for \glspl{ids}.
We evaluated the impact of untargeted and targeted label-flipping attacks on the performance of \gls{fl} models trained on CSE-CIC-IDS2018 and UNSW-NB15 using a standardized feature set to enable the extension of this work.

Our results highlight that
\begin{enumerate*}[(i)]
  \item label-flipping attacks can have a significant impact on the performance of \gls{fl} models, especially targeted ones;
  \item the \gls{asr} is closely related to the number of flipped samples overall, which can be approximated in \gls{iid} settings by the product of \gls{dpr} ($\alpha$) and \gls{mpr} ($\tau$); 
  \item targeted label-flipping attacks strive on well-detected targets, but can be significantly mitigated by the model's generalization capabilities;
  \item mitigation strategies must be adapted to the use case specificities (\eg, constrained environments);
  \item gradient similarity \emph{can} be used to detect label-flipping attacks, but its effectiveness is challenged in heterogeneous settings.
\end{enumerate*}

Yet, there are still many open questions to address.
First, our results can easily be extended with more granular experiments and testing different attack combinations.
On the other hand, while the comparison with existing works seems to corroborate our results, this study calls to be extended to other datasets, feature sets, model architectures, and \gls{fl} aggregation strategies.
Finally, the provided evaluation framework can be used to evaluate the efficiency of existing countermeasures, or to develop new ones.
We strongly believe that this work is a first step towards better evaluation of \gls{fl} models and aggregation strategies in the context of intrusion detection.
In the next chapter, we will build on this work's findings in heterogeneous settings, and propose a novel approach to detect and mitigate low-quality contributions in heterogeneous \gls{fids} environments.


