\usepackage{bookmark}  % allow chapters outside of parts the toc
\usepackage{etoc}  % for local table of contents inside chapters
\usepackage{import} % for importing chapters with subdirectories
\usepackage[
  acronyms,
]{glossaries}
\usepackage{enumerate}
\usepackage[inline,shortlabels]{enumitem} % for custom enumeration labels
\usepackage{lipsum} % for dummy text
\usepackage{cleveref} % for clever references
\usepackage{threeparttable}
\usepackage{booktabs}
\usepackage{multirow}
\usepackage{wasysym}
\usepackage[
  labelfont=bf,
  skip=0.7\baselineskip,
]{caption}

\glsdisablehyper

\setlength{\headheight}{13.59999pt}

%% BIBLIOGRAPHY

\usepackage[
  hyperref,
  backend=biber,
  maxcitenames=1,
  uniquelist=true,
  % https://www.overleaf.com/learn/latex/Biblatex_bibliography_styles
  style=alphabetic
]{biblatex}

\renewcommand*{\newunitpunct}{\addcomma\space} % Virgule entre les parties d'une reference (merci a Josquin Debaz)

%\DeclareFieldFormat[article]{volume}{\textbf{#1}}  %Le numero de volume en gras
\DeclareFieldFormat[article]{number}{\textit{#1}} %Le numero dans le volume en italique
%\DeclareFieldFormat{pages}{page(s): #1} % page(s) en toutes lettres, si on veut...

% Locutions latines en italique (comme ibid, loc.cit. , etc.) Merci \`{a} Josquin Debaz
%\renewcommand{\mkibid}[1]{\emph{#1}}

% Et pour mettre le in en italique dans la ref\'{e}rence biblio (merci encore \`{a} Josquin Debaz)
\DefineBibliographyStrings{english}{%
        in = {\emph{in}}%
}


%% RESEARCH QUESTIONS

\usepackage{amsthm}

%\theoremstyle{plain}
\newtheorem{innerRQ}{RQ}
\crefname{innerRQ}{RQ}{RQs}
\Crefname{innerRQ}{RQ}{RQs}

\newenvironment{RQ}[1]
  {\renewcommand\theinnerRQ{#1}\innerRQ}
  {\endinnerRQ}

\def\changemargin#1#2{\list{}{\rightmargin#2\leftmargin#1}\item[]}
\let\endchangemargin=\endlist 

%% RESEARCH QUERIES

\newlist{queries}{enumerate}{10}
\setlist[queries]{(a)}
\crefname{queriesi}{query}{queries}
\Crefname{queriesi}{Query}{Queries}