\section{Conclusion\label{sec:radar.conclu}}

In this chapter, we introduced \thecontrib, a \acrlong{fl} framework that effectively deals with Byzantine participants, even with heterogeneous data-distributions.
To that end, we introduce a cross-evaluation scheme that allows participants to subjectively estimate their pairwise similarities. 
Based on those measurements, we manage to rebuild the initial participant distribution using hierarchical clustering.
Our results confirm that evaluation metrics can indeed be used to assess similarity between participants, without accessing their datasets nor comparing their models statistically.

We further designed a reputation system based on the cross-evaluation results.
Our reputation system uses the perceived similarity of participants and their cumulated past results to give a score to each participant inside a cluster. 
We are able to validate that the combination of the clustering and reputation system can mitigate all tested Byzantines scenarios, with the single exception of targeted attacks where a majority of Byzantines flip more than 80\% of their labels. 
To the best of our knowledge, this is the first reputation system in \gls{fl} that leverages indirect feedbacks to assess the quality of the participants' contributions.

\thecontrib is the keystone of this thesis, as it addresses the main challenges of \glspl{fids} in an untrusted and heterogeneous environment.
More importantly, it participates in laying out the foundations for the future of \glspl{fids}.
Indeed, its intrinsic qualities, notably the indirect feedbacks and personalized model weighting, makes it a suitable candidate for decentralized architectures.
In this regard, being able to remove the central server dependency is a key step towards a truly decentralized, trustworthy, and privacy-preserving collaborative machine learning framework.
The next chapter will further explore these directions with a discussion on the future of \gls{fids}, and the potential of \thecontrib in this context.
