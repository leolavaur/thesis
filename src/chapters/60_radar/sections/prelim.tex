\section{Preliminaries\label{sec:radar.prelim}}

\Cref{sec:bg.fl.threats} introduced the concept of poisoning attacks in \gls{fl} and mentioned some apt countermeasures, such as Krum~\cite{blanchard_Machinelearningadversaries_2017} or Trimmed Mean~\cite{yin_ByzantineRobustDistributedLearning_2018}.
In this section, we delve deeper into the literature of Byzantine-resilient \gls{fl} and their limitations in \gls{niid} settings.
We also introduce the reader to the concepts of clustering and reputation systems in \gls{fl}, which are used to address the question of data heterogeneity and the reliability of participants' contributions.
Together, these concepts form the basis of our proposed solution, \texttt{RADAR}.


\subsection{Byzantine-resilient Federated Learning\label{sec:radar.prelim.byzantine}}

To assess the quality of the submitted models in \gls{iid} settings, it is common to validate them against a centralized dataset~\cite{cao_FLTrustByzantinerobustFederated_2022}, or against randomly selected distributed datasets~\cite{pejo_QualityInferenceFederated_2023} if they are representative of each other---which is the case with \gls{iid} data partitioning.
Submitted models can also be compared to each other~\cite{blanchard_Machinelearningadversaries_2017,cao_FLTrustByzantinerobustFederated_2022,nguyen_FLAMETamingBackdoors_2022} or with a reference model~\cite{xia_ToFiAlgorithmDefend_2021,zhou_DifferentiallyPrivateFederated_2022}, using distance metrics.
Among these, \texttt{FLAME}~\cite{nguyen_FLAMETamingBackdoors_2022} stands out, as it leverages multiple complementary methods to stop malicious participants: clustering to identify \emph{multiple} groups of attackers, norm-clipping to mitigate gradient boosting attacks, and adaptive noising to lessen the impact of outliers.

However, the assumption of \gls{iid} data rarely holds in \gls{fl}, even though its properties facilitate the detection of Byzantine participants.
Indeed, given \gls{niid} settings, \textcite{you_Poisoningattackdetection_2022} show most of these mitigation strategies are inefficient.
These methods rely on a single source of truth that may be known beforehand~\cite{cao_FLTrustByzantinerobustFederated_2022}, or elected among participants~\cite{blanchard_Machinelearningadversaries_2017}.
However, by definition, this single source of truth does not exist in \gls{niid} datasets. 
To circumvent this issue, \texttt{FoolsGold}~\cite{fung_LimitationsFederatedLearning_2020} and \texttt{CONTRA}~\cite{awan_CONTRADefendingPoisoning_2021} assume that sybils share a common goal, and thus produce similar model updates, allowing to distinguish them from benign \gls{niid} participants that present dissimilar contributions. 
Similar participants are classified as sybils using the cosine similarity between gradient updates, and their weight is reduced in the final aggregation. 
Finally, to avoid relying on a single source of truth or making assumptions about the participants' behavior, the server can directly task participants with evaluating models.
\textcite{zhao_ShieldingCollaborativeLearning_2020} aggregate models in sub-models, which are then randomly attributed to multiple clients for evaluation.  


\subsection{Clustered Federated Learning\label{sec:radar.prelim.cluster}}

\Gls{niid} data can also be regarded as heterogeneous data distributions $\mathcal{P}_k$ that are regrouped together, where $\mathcal{P}_k$ is the distribution of the dataset $d_k$.
Following this idea, some works~\cite{briggs_Federatedlearninghierarchical_2020,ouyang_ClusterFLClusteringbasedFederated_2022,ye_PFedSAPersonalizedFederated_2023} try to group participants sharing similarities. 
The purpose of this approach is twofold.
First, from a performance perspective, \gls{niid} settings slow down convergence.
Even if a global minimum is reached, the model might not be optimal for all participants~\cite{kairouz_AdvancesOpenProblems_2021,ouyang_ClusterFLClusteringbasedFederated_2022}.
In addition, considering outliers as poisoned models~\cite{peri_DeepkNNDefense_2020}, one can eliminate thier in the aggregation process.

Since the effective number of clusters is unknown, hierarchical clustering is a common way to create appropriate clusters~\cite{briggs_Federatedlearninghierarchical_2020,ye_PFedSAPersonalizedFederated_2023}. 
Another approach for finding the appropriate number of clusters is dynamic \emph{split-and-merge} clustering~\cite{chen_ZeroKnowledgeClustering_2021}, where the number of clusters is adjusted depending on the distance between the participants' in each cluster.
Finally, \textcite{ouyang_ClusterFLClusteringbasedFederated_2022} propose a clustering algorithm relying on K-means and spectral relaxation to group participants without prior knowledge of the number of clusters.
Contrary to the most of the existing works, they do not use metrics that rely on vector representations of the models (such as cosine similarity, L2 norm, or scalar products).
Rather, they leverage the \gls{kld} to compare the models' probability distributions, which do not require the models to rely on a convex loss function.


\subsection{Reputation systems for Federated Learning\label{sec:radar.prelim.reputation}}

In collaborative applications, reputation systems preemptively assess the ability of participants to perform a task and the quality of its result, based on past interactions.
\Cref{def:rep} provides a formal definition of reputation systems.
In the context of \gls{fl}, they usually have three main applications:
  \begin{enumerate*}[(i)]
      \item client selection;
      \item model weighting and aggregation; and
      \item tracking contribution quality over time.
  \end{enumerate*}


\begin{definitionbox}{Reputation Systems -- \normalfont\textcite{resnick_Reputationsystems_2000}}{rep}
  A reputation system collects, distributes, and aggregates feedback about participants’ past behavior.
  [\dots]
  To operate effectively, reputation systems require at least three properties: 
  \begin{itemize}[\textbullet]
    \item Long-lived entities that inspire an expectation of future interaction;
    \item Capture and distribution of feedback about current interactions (such information must be visible in the future); and
    \item Use of feedback to guide trust decisions.
  \end{itemize}
\end{definitionbox}

The first application, client selection, is used to determine which participants should be included in the training process of the next round~\cite{kang_ReliableFederatedLearning_2020,awan_CONTRADefendingPoisoning_2021,song_ReputationBasedFederatedLearning_2022,tan_ReputationAwareFederatedLearning_2022}.
This is particularly useful in scenarios with constrained resources~\cite{song_ReputationBasedFederatedLearning_2022} and in hybrid architectures (see \Cref{fig:bg.cids.hybrid}) where servers can exchange reputation information about their users~\cite{kang_ReliableFederatedLearning_2020}.
\texttt{CONTRA}~\cite{awan_CONTRADefendingPoisoning_2021} provides an example of such a reputation system for client selection.
By progressively penalizing the participants that propose models similar to each others, and that are thus suspected of being \emph{sybils} (see \Cref{sec:radar.problem,sec:assess.threat}), it leaves room for participants issuing dissimilar models to be selected more often. 
We detail in \Cref{sec:radar.prelim.byzantine} the limits of these types of approaches in practical \gls{niid} settings.  

The second main application is to weight local models during the aggregation process~\cite{wang_FLAREDefendingFederated_2022,wang_ReputationenabledFederatedLearning_2021}: the higher the reputation, the heavier the local model contributes to the aggregated model.
Some will even go so far as to discard contributions when the author's reputation is too low.
\textcite{karimireddy_LearningHistoryByzantine_2021} underline the importance of historical record in robust aggregation: malicious incremental changes can be small enough to be undetected in a single round but still eventually add up enough to poison the global model over the course of multiple round.
Reputation system's ability to track clients' contributions over time~\cite{kang_ReliableFederatedLearning_2020,wang_ReputationenabledFederatedLearning_2021} can be used as a countermeasure to these attacks.

Finally, note that the literature on reputation systems sometimes distinguishes between \emph{reputation} and \emph{trust} systems~\cite{chen_TRMIoTtrustmanagement_2011,zohrafilali_GlobalTrustTrust_2015}.
One of the main differences is the use of indirect feedbacks in reputation systems, wheras trust systems rely on direct evaluation an objective metrics.
Based on this distinction, the reputation is the global perception of a one's trustworthiness in the system, based on the feedback of others~\cite{chen_TRMIoTtrustmanagement_2011}.
