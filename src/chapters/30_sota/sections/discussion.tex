\section{Discussion\label{sec:sota.discuss}}

This section first discusses the limitations of this study, notably the number of selected papers and the methodology used.
We then answer \Cref{rq:sota.open} by identifying the open issues and according research directions, and associate them with recent publications.


\subsection{Limitations of this Study\label{sec:sota.discuss.limits}}

The original review reviewed 22 technical papers about \glspl{fids}, selected using \gls{slr} methodology.
This ensures that the selected papers are representative of existing works in this field.
Other surveys in similar but broader fields worked with bigger quantities of papers; 231 in \cite{lo_SystematicLiteratureReview_2021} about \gls{fl}, or 95 in \cite{dacosta_InternetThingssurvey_2019} for \gls{ml}-based \glspl{ids}.
Therefore, all conclusion extracted from the selected works must be put in perspective of the number of analyzed papers.

Furthermore, \gls{slr} methodology guarantees the exhaustive aspect of the selection.
However, relevant papers may have been missed; especially, edge--use-cases and unusual wording can exclude papers from the selection process.
We expect the steps presented in \Cref{sec:sota.methodo} to mitigate this risk, notably snowballing.

Moreover, the selected metrics give insight on the quality of the predictions, and more importantly the comparison between \gls{fids} and local detection, when provided.
As the selected works target different use cases with different objectives, a performance metric-based comparison is less relevant.
Using the same datasets, hardware and network configuration, and coding frameworks, a thorough reimplementation of the reviewed papers could provide significant contributions.

Finally, the selected papers are from 2016 to 2021, and the field of \gls{fids} has been evolving rapidly.
As noted in \Cref{sec:sota.methodo.update}, a significant number of papers were published since the end of data collection of the original review, including new literature reviews (see \Cref{sec:sota.related}).
Yet, the conclusions of this study have provided a solid foundation for the other contributions of this thesis and others in the literature.
Further, most trends and research directions that we identified are still relevant in 2024.


\subsection{Open Issues and Future Directions\label{sec:sota.discuss.open}}

As \gls{fl} is becoming more mature, new research tend focus either on side-aspects like security and privacy \cite{bonawitz_PracticalSecureAggregation_2017,fung_LimitationsFederatedLearning_2020,dong_EaSTFLyEfficientsecure_2020,nguyen_PoisoningAttacksFederated_2020b,mothukuri_surveysecurityprivacy_2021} or on its application to a specific use case, as do the works selected in this survey.
This section reviews open questions identified by literature, and the proposed according research directions.
Additionally, for each identified open issue, we provide relevant publications that have been published since the original review to complement the discussion.
Some of these issues depend on works from other related fields, such as \gls{ml} for performance or \gls{fl} for scalability.
However, the specificities of \glspl{fids} require dedicated research.
Especially, the topics of security, trust, and heterogeneity are critical for a collaborative security use case.


\paragraph{Performance}

Like any detection system, \glspl{fids} are looking for an absolute performance: a system with a perfect classification score, producing no false positives or negatives.
To this end, several research directions have been identified by the selected works, such as the use of \glspl{gan}~\cite{schneble_Attackdetectionusing_2019} or the improvement of feature selection as input to the model~\cite{sun_AdaptiveIntrusionDetection_2021}.
More generally, there is a need for a better understanding of the impact of hyper- and meta-parameters on the performance of the system.
This is especially true for \gls{fl}, where the aggregation process can be seen as an optimization problem in itself~\cite{charles_ConvergenceAccuracyTradeOffs_2021}; a problem for which the right parameters need to be inferred.
Both \glspl{gan}~\cite{jin_FLIIDSnovelfederated_2024} and meta-learning for local data-sampling~\cite{he_6GenabledConsumerElectronics_2023} have been reviewed as potential solutions to this problem.


\paragraph{Adaptability}

Constrained environments like low-bandwidth networks, or low-powered devices, may also impact the ability of \gls{fl} to provide detection in a timely fashion (\Cref{chap:background}).
Since the security of constrained devices is a growing concern, the selected works identify relevant research directions in this area, such as implementing compression algorithms~\cite{fan_IoTDefenderFederatedTransfer_2020} or globally reducing the number of computation rounds~\cite{rahman_InternetThingsIntrusion_2020}.
Moreover, as time goes by, the training data can be easily become outdated.
Updating strategies need to be studied to provide accurate results as time goes~\cite{fan_IoTDefenderFederatedTransfer_2020}, and adapt to changes in the traffic behavior \cite{qin_FederatedLearningBasedNetwork_2021}.
This topic has been especially tackled in \glspl{fids} using incremental learning~\cite{jin_FederatedIncrementalLearning_2023}.


\paragraph{Scalability}

Distributed systems such as FL are often used to cope with resource limitations, especially in terms of computation and bandwidth.
However, as pointed out by several selected works, FL faces limitations when dealing with too many clients~\cite{rathore_BlockSecIoTNetBlockchainbaseddecentralized_2019,fan_IoTDefenderFederatedTransfer_2020}.
Therefore, \glspl{fids} require further research regarding client selection: \mbox{performance-,} \mbox{time-,} or reputation-based~\cite{cunhaneto_FedSBSFederatedLearningparticipantselection_2024}.
Moreover, in massively distributed federations, the aggregation process can become a bottleneck.
In such settings, researchers and practitioners might consider using hierarchical aggregation or even complete decentralization of the system.
A few decentralized \glspl{fids} approaches have been proposed since~\cite{friha_2DFIDSDecentralizeddifferentially_2023}.


\paragraph{Heterogeneity and Transferability}

The approaches presented in the initial review mostly consider that all local models share the same architecture and hyper-parameters, use data from the same domain, and that all clients possess similar resources.
These limitations hurdle convergence, and more generally make current \glspl{fids} less versatile and transferable.
Hence, open issues include allowing the federation of cross-domain clients~\cite{li_DeepFedFederatedDeep_2020}.
As pointed out in \Cref{sec:sota.quali.data}, the features selected for model training have to be applicable to multiple environments.
Transfer learning~\cite{saha_FederatedTransferLearning_2021,shen_DistributedMachineLearning_2020} and its federated variant \gls{ftl}~\cite{chen_IntrusionDetectionWireless_2020,fan_IoTDefenderFederatedTransfer_2020} have been applied to similar domains in the past, and might also represent a favorable direction for future research in terms of adaptability.
Since the submission of this study, multiple papers~\cite{otoum_FederatedTransferLearningBased_2021,khoa_DeepTransferLearning_2021,cheng_FederatedTransferLearning_2022} have been published in this direction.


\paragraph{Security and Privacy}

The broad attack surface of \gls{fl} directly applies to \glspl{fids}, raising concerns about poisoning, inference, or model extraction.
The selected works already address some of these issues by leveraging homomorphic encryption to secure the aggregation process~\cite{li_DeepFedFederatedDeep_2020,li_DistributedNetworkIntrusion_2020}. 
Others identify this aspect as potential future works~\cite{chen_IntrusionDetectionWireless_2020}, with countermeasures like \gls{mpc} or \gls{dp}.
Furthermore, as \gls{ml}, and especially \gls{dl}, lacks explainability, the content of a model is difficult to infer. 
It complicates the detection of poisoning attacks, as it is hard to distinguish between a model that has been poisoned and one that has been trained on a different dataset.
Poisoning has recieved a lot of attention in the literature of \gls{fl}~\cite{fung_LimitationsFederatedLearning_2020,nguyen_PoisoningAttacksFederated_2020b,nguyen_FLAMETamingBackdoors_2022}, and some works have been published in the context of intrusion detection too~\cite{yang_Dependablefederatedlearning_2023,merzouk_Parameterizingpoisoningattacks_2023}.


\paragraph{Trust and Reputation}

Following the same line of thought, the trustworthiness of the participants is a critical aspect of \glspl{fids}.
Malicious participants can indeed impact the model and the detection process.
More generally, the quality of the participants' contributions must be controlled to ensure the quality of the aggregated model.
\textcite{zhang_BlockchainbasedFederatedLearning_2020} identify assessing the trustworthiness of the participants as a future research direction.
Inspiration should be taken from the state of the art of collaboration systems and information-sharing platforms, which address problems such as trust or reputations~\cite{wagner_Cyberthreatintelligence_2019a,skopik_problemsharedproblem_2016}, which are relevant for \gls{fids}.
Since the submission of this study, works have been published on the topic of trust via client selection~\cite{cunhaneto_FedSBSFederatedLearningparticipantselection_2024}.
\Cref{chap:radar} presents a solution to this issue, by providing a trust-based model weighting mechanisms to ensure the quality of the aggregated models~\cite{lavaur_radar_2024}.


\paragraph{Self-defense and self-healing}

As highlighted in \Cref{sec:sota.quali.defense}, current research on \gls{fids} is focused on intrusion detection and attack classification.
Mitigation is barely represented in the literature~\cite{rathore_BlockSecIoTNetBlockchainbaseddecentralized_2019}.
However, technologies like \gls{sdn} offer quick mitigation capabilities, and recent works study the effectiveness of such defense mechanisms \cite{bhunia_Dynamicattackdetection_2017,singh_DetectionmitigationDDoS_2020}.
New emerging applications like self-defense and self-healing systems could benefit from \gls{fids} and other \gls{fl}-based technologies.
A handful of works have been published on the topic of attack mitigation and reaction~\cite{panagoda_ApplicationFederatedLearning_2022,decaldasfilho_BotnetDetectionMitigation_2023,phan_FEARFederatedCyberAttack_2022}, corroborating our survey's findings.


\paragraph{Evaluation}

Finally, the topic of evaluation raise two major issues in the selected works.
First, reproducibility is a major concern, as few are the works that provide the code or the datasets used for the experiments.
This is a common issue in the field of \gls{ml}, which has long been criticized for its lack of reproducibility~\cite{bajpai_ChallengesReproducibility_2017,arp_Dosdonts_2022}.
The same issue is present in the field of \glspl{fids}, as most of the selected works do not provide enough information to reproduce the experiments.
Some do not even disclose the datasets they used, such as \citeauthor{nguyen_DIoTFederatedSelflearning_2019} for DÏoT~\cite{nguyen_DIoTFederatedSelflearning_2019}.
Further, existing public datasets are not representative of \glspl{fids} deployment environments.
Indeed, they are often datasets produced for traditional \gls{ml}-based \glspl{ids}, but split for federated purposes.
This makes most of the literature biased, as all samples are related to the same original event.
Recent works have tried to partially address this issue by providing standardized datasets~\cite{sarhan_StandardFeatureSet_2022} or dedicated ones~\cite{ferrag_EdgeIIoTsetNewComprehensive_2022}, although the problem remains unsolved.



