\section{Related Work\label{sec:sota.related}}

At the time of writing this literature review, the literature on \gls{fl} for \gls{ids} was still scarce.
Only a handful of reviews had been published on the topic~\cite{alazab_FederatedLearningCybersecurity_2021,agrawal_FederatedLearningintrusion_2022,campos_EvaluatingFederatedLearning_2022}.
Therefore, we extended our search of related works to related topics that were susceptible to share similar challenges or conclusions.
This extended selection can be divided into three main categories: 
\begin{enumerate*}[(a)]
  \item security information sharing,
  \item intrusion detection, and
  \item collaborative \gls{ml}.
\end{enumerate*}
\Cref{tbl:sota.related} provides a summary of this selection, grouped by topic and sorted by publication date.
In addition to the initial selection, we also included more recent surveys on the topic~\cite{fedorchenko_ComparativeReviewIntrusion_2022, ghimire_RecentAdvancesFederated_2022, ismaila_ReviewApproachesFederated_2024}, whose number highlights the massive interest in the community.


\begin{table}
  \caption[
    Related literature reviews, their topics, contributions, and number of citations.
  ]{
    \emph{Related literature reviews, their topics, contributions, and number of citations according to Google Scholar (Apr.
    2024).}
    Works marked $\ast$ were originally available as preprints, and were only published afterward.
    Works marked $\ddagger$ are added for the sake of completeness, but were not included in the initial selection.
    \label{tbl:sota.related}
  }
    
  \centering
  \resizebox{\textwidth}{!}{%
    
\newcommand*\rotatetitle[1]{\hbox to1.5em{\hss\rotatebox[origin=br]{45}{{\footnotesize #1}}}} % kept to rotate the bottom row if needed
\newcommand*\yes{{\footnotesize\CIRCLE}}
\newcommand*\partly{{\footnotesize\LEFTcircle}}
\newcommand*\nop{{\footnotesize\Circle}}


\newlength{\domaincollength}
\setlength{\domaincollength}{.18\textwidth}

\begin{threeparttable}

    \begin{tabular}{p{\domaincollength}llc@{}c@{}c@{}c@{}c@{}c@{}crr}
        \toprule % ----------------------------------------------------------------
        \textbf{Domain}                                                 & \textbf{Year} & \textbf{Authors}                                                      & \multicolumn{7}{c}{\textbf{Contributions}}                                                                                                                                                                                                                       & \multicolumn{1}{c}{\textbf{Cited}}  & \textbf{Ref.}                                            \\
        \midrule % ----------------------------------------------------------------
        \multirow{4}{\domaincollength}{Security information sharing}    & 2016          & \citeauthor*{skopik_problemsharedproblem_2016}                        & \yes                               & \nop                                & \nop                   & \nop                                 & \nop                                 & \yes                              & \nop                                       & 291                                 & \cite{skopik_problemsharedproblem_2016}                  \\ 
                                                                        & 2018          & \citeauthor*{tounsi_surveytechnicalthreat_2018}                       & \yes                               & \yes                                & \nop                   & \nop                                 & \nop                                 & \yes                              & \nop                                       & 448                                 & \cite{tounsi_surveytechnicalthreat_2018}                 \\
                                                                        & 2019          & \citeauthor*{wagner_Cyberthreatintelligence_2019a}                     & \yes                               & \yes                                & \nop                   & \nop                                 & \nop                                 & \yes                              & \nop                                       & 240                                 & \cite{wagner_Cyberthreatintelligence_2019a}               \\
                                                                        & 2019          & \citeauthor*{pala_InformationSharingCybersecurity_2019}               & \yes                               & \yes                                & \nop                   & \yes                                 & \nop                                 & \yes                              & \nop                                       & 63                                  & \cite{pala_InformationSharingCybersecurity_2019}         \\
        \cmidrule(l){2-12} %        ----------------------------------------------
        \multirow{4}{\domaincollength}{ML for intrusion detection}      & 2016          & \citeauthor*{buczak_SurveyDataMining_2016}                            & \yes                               & \nop                                & \nop                   & \nop                                 & \nop                                 & \partly                           & \nop                                       & 3105                                & \cite{buczak_SurveyDataMining_2016}                      \\ 
                                                                        & 2018          & \citeauthor*{meng_WhenIntrusionDetection_2018}                        & \yes                               & \nop                                & \nop                   & \nop                                 & \nop                                 & \yes                              & \nop                                       & 562                                 & \cite{meng_WhenIntrusionDetection_2018}                  \\ 
                                                                        & 2019          & \citeauthor*{chaabouni_NetworkIntrusionDetection_2019}                & \yes                               & \nop                                & \partly                & \nop                                 & \nop                                 & \yes                              & \nop                                       & 790                                 & \cite{chaabouni_NetworkIntrusionDetection_2019}          \\
                                                                        & 2019          & \citeauthor*{dacosta_InternetThingssurvey_2019}                       & \yes                               & \nop                                & \nop                   & \nop                                 & \nop                                 & \yes                              & \nop                                       & 492                                 & \cite{dacosta_InternetThingssurvey_2019}                 \\
        \cmidrule(l){2-12} %        ----------------------------------------------
        \multirow{2}{\domaincollength}{Collaborative detection}         & 2010          & \citeauthor*{zhou_surveycoordinatedattacks_2010}                      & \yes                               & \nop                                & \nop                   & \nop                                 & \nop                                 & \yes                              & \nop                                       & 517                                 & \cite{zhou_surveycoordinatedattacks_2010}                \\ 
                                                                        & 2015          & \citeauthor*{vasilomanolakis_TaxonomySurveyCollaborative_2015}        & \yes                               & \nop                                & \yes                   & \nop                                 & \nop                                 & \yes                              & \nop                                       & 379                                 & \cite{vasilomanolakis_TaxonomySurveyCollaborative_2015}  \\
        \cmidrule(l){2-12} %        ----------------------------------------------
        \multirow{5}{\domaincollength}{Federated learning}              & 2020          & \citeauthor*{aledhari_FederatedLearningSurvey_2020}                   & \yes                               & \nop                                & \nop                   & \nop                                 & \nop                                 & \nop                              & \nop                                       & 517                                 & \cite{aledhari_FederatedLearningSurvey_2020}             \\
                                                                        & 2020          & \citeauthor*{lyu_ThreatsFederatedLearning_2020} $\ast$                & \yes                               & \nop                                & \nop                   & \nop                                 & \nop                                 & \yes                              & \nop                                       & 436                          & \cite{lyu_ThreatsFederatedLearning_2020}                 \\ 
                                                                        & 2020          & \citeauthor*{shen_DistributedMachineLearning_2020}                    & \yes                               & \nop                                & \nop                   & \nop                                 & \nop                                 & \yes                              & \nop                                       & 69                                  & \cite{shen_DistributedMachineLearning_2020}              \\ 
                                                                        & 2021          & \citeauthor*{mothukuri_FederatedLearningbasedAnomaly_2021}            & \yes                               & \nop                                & \yes                   & \nop                                 & \nop                                 & \yes                              & \nop                                       & 376                                 & \cite{mothukuri_FederatedLearningbasedAnomaly_2021}      \\
                                                                        & 2021          & \citeauthor*{lo_SystematicLiteratureReview_2021}                      & \yes                               & \yes                                & \nop                   & \nop                                 & \nop                                 & \yes                              & \yes                                       & 158                                 & \cite{lo_SystematicLiteratureReview_2021}                \\
        \cmidrule(l){2-12} %        ----------------------------------------------
        \multirow{4}{\domaincollength}{FL for intrusion detection}      & 2021          & \citeauthor*{agrawal_FederatedLearningintrusion_2022} $\ast$          & \yes                               & \nop                                & \nop                   & \nop                                 & \nop                                 & \yes                              & \nop                                       & 142                          & \cite{agrawal_FederatedLearningintrusion_2022}           \\
                                                                        & 2021          & \citeauthor*{alazab_FederatedLearningCybersecurity_2021}              & \yes                               & \nop                                & \nop                   & \nop                                 & \nop                                 & \yes                              & \nop                                       & 158                                 & \cite{alazab_FederatedLearningCybersecurity_2021}        \\ 
                                                                        & 2021          & \citeauthor*{campos_EvaluatingFederatedLearning_2022} $\ast$          & \yes                               & \nop                                & \nop                   & \nop                                 & \yes                                 & \yes                              & \nop                                       & 123                               & \cite{campos_EvaluatingFederatedLearning_2022}           \\ 
                                                                        & \textbf{2022} & \textbf{\citeauthor*{lavaur_tnsm_2022}}    & \yes                               & \yes                                & \yes                   & \yes                                 & \nop                                 & \yes                              & \yes                                       & 22                                  & \cite{lavaur_tnsm_2022}       \\ 
                                                                        & 2022          & \citeauthor*{fedorchenko_ComparativeReviewIntrusion_2022} $\ddagger$  & \partly                            & \nop                                & \nop                   & \nop                                 & \nop                                 & \nop                              & \nop                                       & 22                                  & \cite{fedorchenko_ComparativeReviewIntrusion_2022}       \\
                                                                        & 2022          & \citeauthor*{ghimire_RecentAdvancesFederated_2022} $\ddagger$         & \yes                               & \nop                                & \nop                   & \nop                                 & \nop                                 & \yes                              & \nop                                       & 208                                 & \cite{ghimire_RecentAdvancesFederated_2022}              \\
                                                                        & 2024          & \citeauthor*{ismaila_ReviewApproachesFederated_2024} $\ddagger$       & \yes                               & \yes                                & \nop                   & \nop                                 & \nop                                 & \yes                              & \yes                                       & 0                                   & \cite{ismaila_ReviewApproachesFederated_2024}            \\
        \cmidrule(l){1-3}\cmidrule(l){11-12}
                                                                        &               &                                                                       & \rotatetitle{Qualitative analysis} & \rotatetitle{Quantitative analysis} & \rotatetitle{Taxonomy} & \rotatetitle{Reference architecture} & \rotatetitle{Performance evaluation} & \rotatetitle{Research directions} & \rotatetitle{Systematic Literature Review} &                                     &                                                          \\
    \end{tabular}

    \begin{tablenotes}
        \item \hfil\yes\hspace{.1em} covers topic; \partly\hspace{.1em} partly addresses topic; \nop\hspace{.1em} does not cover topic.\hfil
    \end{tablenotes}

\end{threeparttable}


% \begin{table*}[]
%   \caption{
%     Related works, their topics, contributions, and number of citations according to Google Scholar -- Oct. 2021.
%     \label{tbl:sota.related}
%   }
%   \centering
%   \resizebox{.9\textwidth}{!}{% %used to resize the tikz figure



%       \newcommand*\rott[1]{\hbox to1.5em{\hss\rotatebox[origin=br]{-50}{#1}}}
%       \newcommand*\rotatetitle[1]{\hbox to1.5em{\hss\rotatebox[origin=tr]{20}{#1}}} % kept to rotate the bottom row if needed
%       \newcommand*\rotl[1]{\hbox to1em{\hss\rotatebox{90}{#1}}}
%       \newcommand*\rotb[1]{#1}
%       \newcommand*\yes{\CIRCLE}
%       \newcommand*\partly{\LEFTcircle}
%       \newcommand*\nop{\Circle}
%       \begin{threeparttable}
%           \begin{tabular}{cclc@{}c@{}c@{}c@{}c@{}cr}
%               \textbf{Domain}                                                                                                             & \textbf{Year} & \textbf{Reference}             & \rott{Qualitative literature review} & \rott{Quantitative literature review} & \rott{Taxonomy} & \rott{Reference architecture} & \rott{Performance evaluation} & \rott{Research directions} & \multicolumn{1}{c}{\textbf{Cited}}  \\ \midrule
%               \multirow{4}{*}{Sharing---\labelcref{topic:share}}                                                                          & 2016          & \textcite{Skopik2016}             & \yes                                 & \nop                                   & \nop             & \nop                           & \nop                           & \yes                       & 170                                 \\ 
%                                                                                                                                           & 2018          & \textcite{Tounsi2018}             & \yes                                 & \yes                                  & \nop             & \nop                           & \nop                           & \yes                       & 181                                 \\
%                                                                                                                                           & 2019          & \textcite{Wagner2019}             & \yes                                 & \yes                                  & \nop             & \nop                           & \nop                           & \yes                       & 45                                  \\
%                                                                                                                                           & 2019          & \textcite{Pala2019}               & \yes                                 & \yes                                  & \nop             & \yes                          & \nop                           & \yes                       & 13                                  \\        \cmidrule(l){2-10} %        ----------------------------------------------
%               \multirow{4}{*}{Detection---\labelcref{topic:detect}}                                                                       & 2016          & \textcite{Buczak2016}             & \yes                                 & \nop                                   & \nop             & \nop                           & \nop                           & \partly                    & 1749                                \\ 
%                                                                                                                                           & 2018          & \textcite{Meng2018}               & \yes                                 & \nop                                   & \nop             & \nop                           & \nop                           & \yes                       & 338                                 \\ 
%                                                                                                                                           & 2019          & \textcite{Chaabouni2019}          & \yes                                 & \nop                                   & \partly         & \nop                           & \nop                           & \yes                       & 246                                 \\
%                                                                                                                                           & 2019          & \textcite{DaCosta2019}            & \yes                                 & \nop                                   & \nop             & \nop                           & \nop                           & \yes                       & 152                                 \\ \cmidrule(l){2-20}
%               \multirow{2}{*}{\begin{tabular}{@{}c@{}}Collaborative \\ detection---\labelcref{topic:detect,topic:collab_ml}\end{tabular}} & 2010          & \textcite{Zhou2010}               & \yes                                 & \nop                                   & \nop             & \nop                           & \nop                           & \yes                       & 474                                 \\ 
%                                                                                                                                           & 2015          & \textcite{Vasilomanolakis2015}    & \yes                                 & \nop                                   & \yes            & \nop                           & \nop                           & \yes                       & 270                                 \\ \cmidrule(l){2-20}
%               \multirow{5}{*}{FL---\labelcref{topic:collab_ml}}                                                                           & 2020          & \textcite{Aledhari2020}           & \yes                                 & \nop                                   & \nop             & \nop                           & \nop                           & \nop                        & 83                                  \\
%                                                                                                                                           & 2020          & \textcite{Lyu2020}                & \yes                                 & \nop                                   & \nop             & \nop                           & \nop                           & \yes                       & 101                                 \\ 
%                                                                                                                                           & 2020          & \textcite{Shen2020}               & \yes                                 & \nop                                   & \nop             & \nop                           & \nop                           & \yes                       & 4                                   \\ 
%                                                                                                                                           & 2021          & \textcite{Mothukuri2021}          & \yes                                 & \nop                                   & \yes            & \nop                           & \nop                           & \yes                       & 81                                  \\
%                                                                                                                                           & 2021          & \textcite{Lo2021}                 & \yes                                 & \yes                                  & \nop             & \nop                           & \nop                           & \yes                       & 18                                  \\ \cmidrule(l){2-20}
%               \multirow{4}{*}{FIDS}                                                                                                       & 2021          & \textcite{Agrawal2021}            & \yes                                 & \nop                                   & \nop             & \nop                           & \nop                           & \yes                       & 1                                   \\
%                                                                                                                                           & 2021          & \textcite{Alazab2021}             & \yes                                 & \nop                                   & \nop             & \nop                           & \nop                           & \yes                       & 0                                   \\ 
%                                                                                                                                           & 2021          & \textcite{Campos2021}             & \yes                                 & \nop                                   & \nop             & \nop                           & \yes                          & \yes                       & 0                                   \\ 
%                                                                                                                                           & 2022          & \textbf{Lavaur \emph{et al.}}  & \yes                                 & \yes                                  & \yes            & \yes                          & \nop                           & \yes                       & -                                   \\ \midrule
%                                                                                                                                           &               &                                & \multicolumn{6}{c}{Contributions}                                                                                                                                                           &                                     \\ 
%           \end{tabular}
%           \begin{tablenotes}
%               \item \hfil\yes\hspace{.1em} covers topic; \partly\hspace{.1em} partly addresses topic; \nop\hspace{.1em} does not cover topic;\hfil
%           \end{tablenotes}
%       \end{threeparttable}
%   }
% \end{table*}

% 
% <!-- Markdown table kept for reference
% | Reference             | Topics                                                       | Cited |
% |-----------------------|--------------------------------------------------------------|------:|
% | Chantzios et al. 2019 | \gls{cti}; information sharing                               |     1 |
% | ENISA 2017            | information sharing; standardization                         |   N/A |
% | Wagner et al. 2019    | \gls{apt}; \gls{cti}; information sharing; trust             |    15 |
% | Pala et al. 2019      | security; information sharing                                |     6 |
% | Tounsi et al. 2018    | cyber crime; \gls{ioc}; \gls{cti}; threat sharing            |   111 |
% | Skopik et al. 2016    | incident reporting; security; information sharing            |   150 |
% | Lin et al. 2018       | \gls{iot}; security; privacy; fog/edge-computing             |  1300 |
% | Sha et al. 2020       | \gls{iot}; privacy; detection; fog/edge-computing            |    20 |
% | Neshenko et al. 2019  | \gls{iot}; security; vulnerabilities                         |   139 |
% | Sengupta et al. 2020  | \gls{iiot}; security; privacy; blockchain;                   |    76 |
% | Panchal et al. 2018   | \gls{iiot}; security                                         |    17 |
% | Meng et al. 2018      | blockchain; information sharing; detection; trust management |   236 |
% | Jiang et al. 2019     | \gls{ics}; \gls{ml}; detection                               |   N/A |
% | Aledhari et al. 2020  | \gls{fl}; collaborative \gls{ai}; security; privacy           |    20 |
% | Mothukuri et al. 2021 | \gls{ai}; \gls{fl}; security; privacy                         |     7 |
% | Lyu et al. 2020       | \gls{fl}; \gls{ai}; privacy preservation;                     |    30 |
% | Decker 1987           | \gls{ai}; distributed processing; problem-solving            |   318 |
% | Radulescu et al. 2019 | \gls{momas}; \gls{dm}; \gls{rl}                              |    12 |
% : Reference table -->

  }
\end{table}


Common issues of collaborative systems, such as the need for trust, privacy, and security, can also apply to \gls{fl}-based collaboration systems.
Therefore, we include four surveys~\cite{skopik_problemsharedproblem_2016,tounsi_surveytechnicalthreat_2018,wagner_Cyberthreatintelligence_2019a,pala_InformationSharingCybersecurity_2019} where the authors discuss the challenges and opportunities of sharing security-related information.
They highlight the need for standardization, automation, and incentives, to achieve efficient and effective collaboration.
The topic of trust is a clearly identified challenge in these works~\cite{wagner_CyberThreatIntelligence_2019,tounsi_surveytechnicalthreat_2018}.
The present study rather focuses on \gls{fl} as a technical mean for collaboration, but such as trust or incentives are also relevant in this context.

Because \gls{ml}-based \gls{ids} can be considered as a key component of \gls{fids}, we review existing surveys on the topic~\cite{buczak_SurveyDataMining_2016,meng_WhenIntrusionDetection_2018,chaabouni_NetworkIntrusionDetection_2019,dacosta_InternetThingssurvey_2019}.
These work cover a wide range of solutions, from traditional \gls{ml} (\gls{svm}, \gls{dt} and \gls{rf}, among others) to more recent approaches, such as deep learning, the latter being overrepresented in the literature of \glspl{fids}.
They also provide a good overview of the existing datasets and evaluation metrics, which can be useful for the evaluation of \gls{fl}-based \gls{ids}.
However, as noted in \Cref{sec:sota.discuss.open}, typical \gls{ids} datasets present limitations that can hinder the evaluation of \gls{fl}-based \gls{ids}.

\Gls{fl} is obviously another critical aspect of \glspl{fids}.
Consequently, related works include surveys on the collaborative aspects of \gls{ml} (b) and \gls{fl}~\cite{aledhari_FederatedLearningSurvey_2020,lo_SystematicLiteratureReview_2021}.
They discuss \gls{fl} approaches to work with distributed architectures.
The security of \gls{fl} is also heavily reviewed by~\cite{shen_DistributedMachineLearning_2020,lyu_ThreatsFederatedLearning_2020,mothukuri_surveysecurityprivacy_2021}.
They identify security threats like communication bottleneck, poisoning, and \gls{ddos} attacks, that could endanger \gls{fl}-based systems.
While the \gls{ids} use case can be seen as an application of \gls{fl}, we argue that it raises specific concerns in terms of privacy, latency, and adaptability.

\textcite{zhou_surveycoordinatedattacks_2010,vasilomanolakis_TaxonomySurveyCollaborative_2015} survey the evolution of \gls{cids}---at the merge of intrusion
detection (b) and collaborative \gls{ml} (c).
Their works are however older and thus, cannot offer a comprehensive view of \gls{cids}, as \gls{fl}-based approaches did not exist at the time of their writing.
Hence, the authors focus on collaboration in the sense of \emph{detection}$+$\emph{correlation}, whereas the analysis presented in this chapter (\Cref{sec:sota.quali}) surveys the use of \gls{fl} in \glspl{ids}.

In addition to the above, recent work (\ie, contemporary to the writing of the initial study) have reviewed the use of \gls{fl} for intrusion
detection~\cite{agrawal_FederatedLearningintrusion_2022,campos_EvaluatingFederatedLearning_2022,alazab_FederatedLearningCybersecurity_2021}.
\textcite{alazab_FederatedLearningCybersecurity_2021} address the wider topic of
\gls{fl} for cybersecurity, which only includes intrusion detection as an application.
Their paper
is explanatory and provides an overview of \gls{fl} applications in information security.
Like this
work, \textcite{agrawal_FederatedLearningintrusion_2022} focus on \glspl{fids}, but have different methodology.
The authors list
existing \glspl{fids} and detail their approaches, and identify open issues.
On the other hand,
\textcite{campos_EvaluatingFederatedLearning_2022} review a subset of \glspl{fids} by focusing on \gls{iot} use case, and the
impact of non-\acrshort{iid} (\acrlong{iid}) data on performance.
While all identify challenges and
research directions, this work also performs quantitative (\Cref{sec:sota.quanti}) and
qualitative (\Cref{sec:sota.quali}) analyses of existing \glspl{fids}, and extracts reference
architecture and taxonomy.
The existence of these papers emphasizes the importance and relevance of
\glspl{fids} for the research community.

The more recent works on the topic~\cite{fedorchenko_ComparativeReviewIntrusion_2022, ghimire_RecentAdvancesFederated_2022, ismaila_ReviewApproachesFederated_2024} confirm these observations.
The work of \textcite{fedorchenko_ComparativeReviewIntrusion_2022} is of little interest, as it only lists and details existing works with close to no added value.
\textcite{ghimire_RecentAdvancesFederated_2022} provide a more convincing study, closer to the method applied by \textcite{alazab_FederatedLearningCybersecurity_2021}, but with a focus on the \gls{iot}.
Finally, \textcite{ismaila_ReviewApproachesFederated_2024} provide a comprehensive review, with up-to-date literature leveraging the \gls{slr} methodology, but still focuses on the \gls{iot}.
