\selectlanguage{french}
\section{Remerciements en français\label{sec:french}}

Les mots vont manquer pour exprimer l'infinie gratitude que je ressens envers toutes les personnes sans qui ce travail n'aurait pas été possible.
J'ai lu trop de remerciements de thèse pour ne pas être conscient de la difficulté de cet exercice, en particulier pour décrocher l'exhaustivité tant désirée.
Sachant qu'elle ne saura jamais être atteinte, j'espère tout de même ne pas commettre trop d'impairs et que les personnes concernées se reconnaîtront dans ces quelques lignes.

Je tiens tour d'abord à remercier chaleureusement Vincent Nicomette d'avoir accepté de présider mon jury, ainsi que Sonia Ben Mokhtar et Pierre-François Gimenez d'avoir accepté d'en faire partie.
Merci tout particulièrement aux rapporteurs, Anne-Marie Kermarrec (ma \emph{grand-mère} académique) et Éric Totel, d'avoir accepté de lire et juger un manuscrit pas forcément court et reçu durant l'été.
Ces remerciements s'étendent bien sûr aux membres de mon commité de suivi de thèse, qui ont suivi mes travaux et m'ont conseillé à chacune de nos rencontres : Nur Zincir-Heywood, Hanan Lutfiyya et David Espes.
Merci aussi à mes enseignants de l'ENSIBS, en particulier à Vianney, Seb et Jamal, pour m'avoir encouragé à poursuivre vers la recherche et m'avoir donné les bases nécessaires pour y arriver.
Je remercie également Marc-Oliver Pahl, titulaire de la chaire CyberCNI, de m'avoir recruté et encadré, ainsi que tous les partenaires d'avoir participé à financer cette aventure.

Ce travail n'aurait surtout pas été possible sans mon directeur de thèse, Yann Busnel, et mon co-encadrant, Fabien Autrel. 
Merci à Yann pour m'avoir fait confiance et m'avoir laissé une liberté totale dans mes travaux, tout en me guidant et me conseillant à chaque étape, ce malgré la distance et les responsabilités professionnelles toujours plus nombreuses.
Ses intuitions, son expérience et sa bienveillance m'ont mené où je suis aujourd'hui.
Tout ceci, évidemment, sans oublier notre goût partagé pour les liqueurs maltées.
Merci à Fabien pour ce support inconditionnel au quotidien, mais aussi son investissement technique sans lequel certains travaux n'auraient pas vu le jour.
Merci pour la bonne humeur et les blagues à la qualité variable, mais toujours appréciées.
À tous les deux, je vous dois énormément.

Certains doctorants considèrent leur établissement d'accueil comme un lieu de passage, mais le campus de Rennes d'IMT Atlantique et tout particulièrement le département SRCD ont été pour moi beaucoup plus que ça.
J'ai eu ici des discussions si riches et variées qui ont bien dépassé le cadre professionnel et scientifique.
Avec Loutfi de montagne ou de musique et cinéma français si Xavier est à côté, avec Guillaume de cuisine et de poêles De Buyer, avec Jean-Marie de vélo, avec Renzo de philosophie, avec Nico de jeux vidéo\dots{}
Sans oublier Hélène, Laurent, Patrick, Alexander, Léonie, Briac, Baptiste, Alexis, Benoit, Georgios ou Nicolas, et ceux qui portent le département et le campus sur leurs épaules : Sandrine, Catherine, Frank, Vincent et les Delphines.
Mais dans tout ce monde incroyable, je tiens à remercier tout particulièrement ce noyau dur que forment Gégé, Nico et Rom qui m'ont accueilli, conseillé et tant appris, mais aussi emmené dans les pires traquenards de Saint-Hélier.
Merci à tous pour ces moments de partage et de convivialité, je chérirai toujours ces années passées à IMT Atlantique.

Mes remerciements vont aussi bien sûr à mes collègues de travail au sein de l'équipe SOTERN, dans laquelle ne n'ai pas encore mentionné Ahmed, Pierre, Aymen, Gilles, ainsi que tous les non-permanents qui ont contribué à la bonne ambiance et à la productivité du groupe : Loïc, Antoine, Tien, Anh, Mathis, et les petits derniers, Hugo, Constant, Dorian et Gabriel.
Bien que répartie sur deux sites, je suis heureux d'avoir pu travailler avec une équipe aussi riche et humaine.

Des pensées toutes particulières aux collègues doctorants et post-doctorants qui ont partagé tout ou partie de ce voyage avec moi.
Ceux qui montrent que c'est possible, Renzo et Loïc, mes co-bureaux, Gwen et Awaleh, puis Arnol, et enfin Tien et Loïc de nouveau, mais aussi Juliette, Modou, et tous les autres.
Un merci tout particulier à mon cartel de l'ENSIBS infiltré au sein d'IMT Atlantique, Antoine et Pierre-Marie, avec qui j'ai la fierté de former ce trio infernal capable de faire tomber le réseau informatique de l'école.
Pierre-Marie tout particulièrement, merci de m'avoir supporté pendant ces deux ans de collaboration, quitte à passer des soirées à se battre sur des macros \LaTeX{} ou l'organisation du Gitlab de RADAR.
Je suis fier de ce que nous avons accompli ensemble.

Je ne saurais terminer ces remerciements sans mentionner mes amis et ma famille, qui m'ont soutenu et encouragé tout au long de ces années.
Merci à ma \texttt{/(Banana|Sticks) Team/g}, Mathieu, Nathan et Hippo, ainsi qu'aux autres "Copains", Kraft, Popo et Lolo, pour ces moments de détente et de rire qui m'ont permis de naviguer cette période parfois difficile.
Merci à Brieuc, mon frère de c{\oe}ur et ami de toujours, ainsi qu'à sa compagne Clarisse, pour leur présence même à distance, ainsi qu'à tous les autres qui se reconnaîtront.
Merci à mes parents, qui m'ont toujours soutenu dans mes choix et m'ont donné les moyens de réaliser mes ambitions, ainsi qu'à ma s{\oe}ur et tous mes proches qui m'ont encouragé et supporté, même s'ils ne savent pas toujours ce que je fais.
Enfin, je remercie infiniment ma compagne Emma, ma \emph{Mimie}, pour sa patience, son soutien et son amour inconditionnel, sans lesquels je n'aurais jamais eu la force de terminer ce travail.

Merci aussi à tous ceux que je n'aurai pas cité ici, mais qui ont contribué à leur manière à la réalisation de ce doctorat.
À tous, merci.






