\section{Fundamentals of Federated Learning\label{sec:bg.fl}}


% \Gls{fl} can also vary depending on the objectives on the federation.
% \Gls{cdfl} is a federated setting where on-devices models are trained, and then aggregated to be used by the server.
% It is especially useful to learn from user data (\eg from smartphones or wearables) while respecting privacy and trust issues~\cite{Kairouz2021}.
% When clients are organizations, in use cases like network security or fraud detection, we use the term \gls{csfl}~\cite{Kairouz2021}.

% Most applications are using \gls{hfl} \cite{Yang2019}, which is close to distributed learning.
% In the case of \gls{hfl}, the different clients share the same features, but not the same samples.
% Thus, \gls{hfl} particularly copes with \emph{ground-truth} issues (\Cref{chall:ground-truth}) by providing more data for the model to be trained on.

% The other variants of \gls{fl}, namely \gls{vfl} and \gls{ftl}~\cite{Yang2019}, are less represented in the literature of \gls{fids}.
% The choice of a \gls{fl} approach depends on the part of \emph{samples} and \emph{features} that is shared by clients.
% A sample is an individual entry in a dataset.
% Features are measurable characteristics of this entry~\cite{Bishop2006}.
% When storing a dataset as a matrix for processing, each row represents a sample, and each column a feature~\cite{Yang2019}.

% \Gls{hfl} is applicable when clients share features but not samples, which is the case in most reviewed works, as denoted in \Cref{sec:results.quali.fl}.
% \Gls{vfl} is the opposite: clients share samples but have access to different feature spaces.
% An identifier is shared among the samples, so a correlation can be done between the samples of different clients.
% In \gls{ftl}, only a subset of both, features and samples, is shared.
% \Gls{ftl} is often used to transfer the knowledge of a well-trained model to a slightly different use case or context, \eg different networks configurations or device types.

\subsection{Types of Federated Learning\label{sec:bg.fl.types}}
%   - Cross-device VS cross-silo
%   - HFL vs. VFL vs. FTL
%   - Architecture discrepancies


\subsection{The \texttt{FedAvg} Algorithm\label{sec:bg.fl.fedavg}}
%     - Notations


\subsection{The Question of Data Distribution\label{sec:bg.fl.data}}

% iid vs non-iid
% types of non-iid
% how to handle non-iid data
% - algos: FedProx, Fed+
% - data augmentation
% - client-side sampling
% - clustering (dedicated section in chap:radar)


\subsection{Threats against Federated Learning\label{sec:bg.fl.threats}}

% - Threats against FL
%   - Threats summary
%   - Focus on data poisoning
%     - Attacker modeling
%     - Attack vectors
%     - Mitigation strategies