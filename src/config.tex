\usepackage{bookmark}  % allow chapters outside of parts the toc
\usepackage{etoc}  % for local table of contents inside chapters
\usepackage{import} % for importing chapters with subdirectories
\usepackage[
  acronyms,
]{glossaries}
\usepackage{enumerate}
\usepackage[inline,shortlabels]{enumitem} % for custom enumeration labels
\usepackage{lipsum} % for dummy text
\usepackage{cleveref} % for clever references
\usepackage{threeparttable}
\usepackage{booktabs}
\usepackage{multirow}
\usepackage{wasysym}
\usepackage[
  labelfont=bf,
  skip=0.7\baselineskip,
]{caption}
\usepackage{mwe}
\usepackage{layouts}
\usepackage{pgf}
\usepackage{pgfplots}
\usepackage{subcaption}
\usepackage[export]{adjustbox}
\usepackage{float}
\usepackage{tikz}
\usepackage[edges]{forest}
% \usepackage[
%   chapter,
%   outputdir=../build,
% ]{minted}
\usepackage{listings}
\usepackage{tabularx}
\usepackage[table,dvipsnames]{xcolor}

\setlength{\headheight}{13.59999pt}

%% GLOSSARY
\glsdisablehyper
\preto\chapter{\glsresetall}

%% BIBLIOGRAPHY

\usepackage[
  hyperref,
  backend=biber,
  uniquelist=true,
  % https://www.overleaf.com/learn/latex/Biblatex_bibliography_styles
  style=alphabetic,
  sortcites=true,
]{biblatex}

\renewcommand*{\newunitpunct}{\addcomma\space} % Virgule entre les parties d'une reference (merci a Josquin Debaz)

%\DeclareFieldFormat[article]{volume}{\textbf{#1}}  %Le numero de volume en gras
\DeclareFieldFormat[article]{number}{\textit{#1}} %Le numero dans le volume en italique
%\DeclareFieldFormat{pages}{page(s): #1} % page(s) en toutes lettres, si on veut...

% Locutions latines en italique (comme ibid, loc.cit. , etc.) Merci \`{a} Josquin Debaz
%\renewcommand{\mkibid}[1]{\emph{#1}}

% Et pour mettre le in en italique dans la ref\'{e}rence biblio (merci encore \`{a} Josquin Debaz)
\DefineBibliographyStrings{english}{%
  in = {\emph{in}}%
}

% Et al. in italics
% see: https://tex.stackexchange.com/a/40816
\renewbibmacro*{name:andothers}{% Based on name:andothers from biblatex.def
  \ifboolexpr{
    test {\ifnumequal{\value{listcount}}{\value{liststop}}}
    and
    test \ifmorenames
  }
    {\ifnumgreater{\value{liststop}}{1}
       {\finalandcomma}
       {}%
     \andothersdelim\bibstring[\emph]{andothers}}
    {}}


%% RESEARCH QUESTIONS



\newlist{questions}{enumerate}{2}
\setlist[questions,1]{
  label={RQ\arabic*.},
  ref={RQ\arabic*},
  font=\normalshape\bfseries,
  before=\itshape,
  %after=\normalfont\mdseries\normalshape\normalsize,
  labelindent=0pt,
  leftmargin=*,
  align=left,
}
\setlist[questions,2]{
  label={\textbf{\arabic{questionsi}.\alph*.}},
  ref={\thequestionsi\alph*},
  %font=\normalshape,
  before=\normalshape\small,
  %after=\normalfont\mdseries\normalshape\normalsize,
  % font=\normalshape,
}
\crefname{questions}{question}{questions}
\Crefname{questions}{Question}{Questions}
\crefname{questionsii}{question}{questions}
\Crefname{questionsii}{Question}{Questions}

% subquestions
% environment with a parameter to refer to an existing question
\newenvironment{subquestions}[1]
  {\begin{questions}[label={\ref{#1}-\arabic*.}]}
  {\end{questions}}



%% RESEARCH QUERIES

\newlist{queries}{enumerate}{10}
\setlist[queries]{(a)}
\crefname{queriesi}{query}{queries}
\Crefname{queriesi}{Query}{Queries}

%% REFERENCES

\newcommand{\needref}{\textbf{[?]}}

%% TOCs
\newcommand{\localtoc}{
  \vspace{-9ex}
  \localtableofcontents
  \noindent\rule{\textwidth}{0.4pt}
}

%% Debug
\makeatletter
\newcommand{\printfontsize}[1]{{#1\the\dimexpr\f@size pt\relax}}
\makeatother

%% Matplotlib
\def\mathdefault#1{#1}\everymath=\expandafter{\the\everymath\displaystyle}
% \usepackage{fontspec}
% \makeatletter\@ifpackageloaded{underscore}{}{\usepackage[strings]{underscore}}\makeatother  


\makeatletter
\newcommand*{\preventbreak}{\vspace{\baselineskip}\par\nobreak\@afterheading}
\makeatother


%
% Properly spaced abbreviations
% Taken from the CVPR's style package (https://stackoverflow.com/a/39363004)
%
\usepackage{xspace}

% Add a period to the end of an abbreviation unless there's one
% already, then \xspace.
\makeatletter
\DeclareRobustCommand\onedot{\futurelet\@let@token\@onedot}
\def\@onedot{\ifx\@let@token.\else.\null\fi\xspace}
%
\def\eg{\emph{e.g}\onedot} \def\Eg{\emph{E.g}\onedot}
\def\ie{\emph{i.e}\onedot} \def\Ie{\emph{I.e}\onedot}
\def\cf{\emph{c.f}\onedot} \def\Cf{\emph{C.f}\onedot}
\def\etc{\emph{etc}\onedot} \def\vs{\emph{vs}\onedot}
\def\wrt{w.r.t\onedot} \def\dof{d.o.f\onedot}
\def\etal{\emph{et al}\onedot}
\makeatother


\newcommand{\fullcref}[1]{\nameref*{#1} \labelcref{#1}}


% Boxes



\usepackage[skins,theorems]{tcolorbox}

\definecolor{titlebg}{RGB}{85,85,85}
\definecolor{boxbg}{RGB}{232,232,232}

\tcbset{
  center,
  width=0.9\textwidth,
  boxrule=1pt,
  colback=boxbg,
  colframe=titlebg,
  fonttitle=\bfseries\sffamily,
  colbacktitle=titlebg,
}

\newtcolorbox{highlightbox}[2][]{
  enhanced,
  title={#2},
  boxsep=10pt,
  attach boxed title to top center={
    yshift=-\tcboxedtitleheight/2,
    yshifttext=-\tcboxedtitleheight/2,
  },
  boxed title style={
    sharpish corners,
  },
  #1,
}

% Use as: \begin{highlightbox}{Title}{label} ... \end{highlightbox}
%
% \newtcbtheorem
%   [<tcb options>]  % tcolorbox options, like you would use with \newtcolorbox
%   {<name>}         % name of the environment
%   {<display name>} % name of the environment in the output
%   {<options>}      %
%   {<prefix>}       % prefix for the label. A label is automatically created as \label{prefix:#2}

\newtcbtheorem[
  number within=chapter,
  crefname={definition}{definitions},
  Crefname={Definition}{Definitions},
]{definitionbox}{Definition}{}{def}


% listings
\usepackage{newfloat}
\usepackage{inconsolata}
\lstset{basicstyle=\footnotesize\ttfamily,breaklines=true}
\DeclareFloatingEnvironment[fileext=frm,placement={ht},name=Listing,within=chapter]{listing}


%challenge
\usepackage{amsthm}
\theoremstyle{definition}
\newtheorem{challenge}{Challenge}
\crefname{challenge}{challenge}{challenges}
\Crefname{challenge}{Challenge}{Challenges}

% maths
\usepackage{amsmath}
\usepackage{amssymb}
\usepackage{amsfonts}
\usepackage{stmaryrd}

% marks
\usepackage{pifont}
\newcommand{\cmark}{\ding{51}}
\newcommand{\xmark}{\ding{55}}

% algorithms
\usepackage{algorithm}
\usepackage{algpseudocodex}
\counterwithin{algorithm}{chapter}
% Special block definition
% ------------------------------------------------------------------------------
% aglorithmic block definitions (https://github.com/chrmatt/algpseudocodex/issues/3)
\algnewcommand\algorithmicwith{\textbf{with}}%

\makeatletter
\algdef{SE}[WITH]{With}{EndWith}[1]{\algpx@startCodeCommand\algpx@startIndent\algorithmicwith\ #1\ \algorithmicdo}{\algorithmicend\ \algorithmicwith}%
\ifbool{algpx@noEnd}{%
  \algtext*{EndWith}%
  %
  % end indent line after (not before), to get correct y position for multiline text in last command
  \apptocmd{\EndWith}{\algpx@endIndent}{}{}%
}{}%

\pretocmd{\With}{\algpx@endCodeCommand}{}{}

% for end commands that may not be printed, tell endCodeCommand whether we are using noEnd
\ifbool{algpx@noEnd}{%
  \pretocmd{\EndWith}{\algpx@endCodeCommand[1]}{}{}%
}{%
  \pretocmd{\EndWith}{\algpx@endCodeCommand[0]}{}{}%
}%
\makeatother

% bib sort
\DeclareSortingTemplate{ymdnt}{
  \sort{
    \field{presort}
  }
  \sort[final]{
    \field{sortkey}
  }
  \sort[direction=descending]{
    \field{sortyear}
    \field{year}
    \literal{9999}
  }
  \sort[direction=descending]{
    \field[padside=left,padwidth=2,padchar=0]{month}
    \literal{99}
  }
  \sort[direction=descending]{
    \field[padside=left,padwidth=2,padchar=0]{day}
    \literal{99}
  }
  \sort{
    \field{sortname}
    \field{author}
    \field{editor}
    \field{translator}
    \field{sorttitle}
    \field{title}
  }
  \sort{
    \field{sorttitle}
  }
  \sort[direction=descending]{
    \field[padside=left,padwidth=4,padchar=0]{volume}
    \literal{9999}
  }
}

\usepackage{colortbl}
\usepackage{multirow}
\usepackage{makecell,rotating}