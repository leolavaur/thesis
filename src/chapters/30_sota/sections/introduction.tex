\section{Introduction and Motivation\label{sec:sota.intro}}

In the previous chapter, we introduced the concepts of \acrfull{ids} and \acrfull{ml}, the challenges of deploying \acrfullpl{cids}, and why \acrfull{fl} is a promising solution to these challenges. 
This chapter's prime objective is to provide a comprehensive review of how \gls{fl} can be leveraged for intrusion detection purposes, and shed light on the gaps in the literature that are discussed in this thesis.

\paragraph{A recent topic without identity}

Because of the novelty of \gls{fl} in the field of \gls{ids}, the literature on the topic is still scarce.
Only a handful of reviews~\cite{alazab_FederatedLearningCybersecurity_2021,agrawal_FederatedLearningintrusion_2022,campos_EvaluatingFederatedLearning_2022} existed on the topic when we stopped our data collection for this study in late 2021, most of which only as preprint.
While these papers provide a good overview of the existing works, they fail to provide synthesis and extract the core characteristics of the field.
Notably, \emph{what makes \gls{fl} for \gls{ids} different from \gls{fl} for other applications, and what challenges are specific to the field of intrusion detection?}


\paragraph{A systematic approach}

We aim to address this gap as thoroughly and transparently as possible, and leverage the \acrfull{slr} methodology to that end.
This method relies on a structured process to identify, select, and analyze the relevant literature on a given topic.
With explicitly defined research questions and inclusion/exclusion criteria, the \gls{slr} methodology ensures that the review is reproducible and unbiased.
Therefore, we intend to provide a comprehensive overview of the existing literature, and reproducible, evidence-based conclusions on the specificities of \gls{fl} for \gls{ids}.

\paragraph{Content}

The content of this chapter is based on our survey published in \gls{tnsm} in May 2022~\cite{lavaur_EvolutionFederatedLearningbased_2022} and its accompanying extension at the C\&ESAR conference in November 2022~\cite{lavaur_Federatedlearningenabler_2022}.
Because the initial paper was submitted in November 2021, the quantitative analysis has been updated during the writing of this manuscript to include the latest publications on the topic.
The qualitative analysis has also completed to a lesser extent.

\begin{contribs}
  \item The first \gls{slr} on the use of \gls{fl} for \gls{ids}, including qualitative and quantitative analyses of the literature.
  \item A generalization of the selected works as a reference architecture for \glspl{fids}, providing a starting point for future works on the topic.
  \item A taxonomy synthesizing the state of the art of \gls{fids}, providing a framework to analyze and compare existing and upcoming literature.
  \item The identification of the main challenges and opportunities in the field, and a set of research directions to address them.
\end{contribs}

